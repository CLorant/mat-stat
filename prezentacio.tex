\documentclass[aspectratio=169,12pt]{beamer}

\usepackage[magyar]{babel}
\usepackage{t1enc}
\usepackage{graphicx}
\usepackage{amsmath}

\definecolor{craneyellow}{RGB}{233, 187, 61}
\definecolor{craneblue}{RGB}{4, 6, 76}

\usetheme[progressbar=frametitle]{metropolis}
\usecolortheme{crane}
\usefonttheme{professionalfonts}

\hypersetup{
	colorlinks=true,
	linkcolor=craneblue,
	urlcolor=craneyellow
}

\title{Centrális Határeloszlás-tétel (CHT)}
\subtitle{Matematikai Statisztika}
\author{Czibik Lóránt Patrik}
\institute{Miskolci Egyetem}
\date{\today}

\begin{document}
	\begin{frame}
		\titlepage
	\end{frame}
	
	\section{Bevezetés}
	\begin{frame}{Bevezetés}
		\begin{columns}
			\begin{column}{0.6\textwidth}
				\textbf{A Centrális Határeloszlás-tétel lényege:}
				\begin{itemize}
					\item Nagyméretű minták átlaga közelít a normális eloszláshoz.
					\item Fontos a statisztikában és a valószínűségszámításban.
					\item Gyakorlati alkalmazások a különböző tudományterületeken.
				\end{itemize}
			\end{column}
			\begin{column}{0.4\textwidth}
				\begin{figure}
					\includegraphics[width=\textwidth]{img/bevezeto.png}
					\caption{A normális eloszlás közelítése binomiális eloszlásokkal}
				\end{figure}
			\end{column}
		\end{columns}
	\end{frame}
	
	\section{Matematikai alapok}
	\begin{frame}{A Centrális Határeloszlás-tétel definíciója}
		\textbf{Definíció:} Legyenek $\xi_1, \xi_2, \ldots, \xi_n$ független, azonos eloszlású valószínűségi változók, melyek várható értéke $\mu$ és szórása $\sigma$. Ekkor a megfelelően normált összeg:
		\begin{equation}
			Z_n = \frac{\sum_{i=1}^{n} \xi_i - n \mu}{\sigma \sqrt{n}}
		\end{equation}
		$n \to \infty$ esetén standard normális eloszlást követ: $Z_n \sim N(0,1)$.
	\end{frame}
	
	\section{Alkalmazások és példák}
	\begin{frame}{Gyakorlati alkalmazások}
		\begin{itemize}
			\item \textbf{Statisztikai elemzés:} Segít megérteni, hogyan viselkednek az átlagok nagyobb minták esetén, például közvélemény-kutatásokban.
			\item \textbf{Minőségellenőrzés:} Gyárakban segít ellenőrizni, hogy a termékek mérete és minősége egyenletes maradjon.
			\item \textbf{Pénzügy:} Használják annak elemzésére, hogyan változnak az árfolyamok és egyéb pénzügyi mutatók az idő múlásával.
		\end{itemize}
	\end{frame}
	
	\begin{frame}{Gyakorlati példák}
		Magyarázatot ad arra, hogy sok természetes és mesterséges jelenség \textbf{miért követ normális eloszlást:}
		\begin{itemize}
			\item Az emberek magassága
			\item Az emberek napi kalóriabevitele
			\item Egy gyártósoron készült alkatrészek mérete (pl. csavarok hossza)
			\item A vonatok érkezési késései (ez nem vonatkozik a MÁV-ra)
		\end{itemize}
	\end{frame}
	
	\section{R nyelv és használt csomagok}
	\begin{frame}{Az R nyelv}
		Az R egy nyílt forráskódú programozási nyelv statisztikai számításokhoz és adatelemzéshez.
		\vfill
		\begin{block}{Miért használtam}
			\begin{itemize}
				\item Komplex műveletek egyszerűen megvalósíthatóak
				\item Könnyen bővíthető csomagokkal
				\item Korábbi tapasztalat
			\end{itemize}
		\end{block}
	\end{frame}
	
	\begin{frame}{ggplot2 csomag}
		A \texttt{ggplot2} csomag a grafikonok készítésére szolgál, ebben az esetben például hisztogramokhoz és sűrűségfüggvényekhez.
		\vfill
		\begin{block}{Miért használtam}
			\begin{itemize}
				\item A mintaátlagok eloszlásának szemléltetésére hisztogramokkal és sűrűségfüggvényekkel.
				\item A normális eloszlás és az empirikus adatok összehasonlítására.
				\item Lehetővé teszi az ábrák testreszabását (pl. színek, jelmagyarázat, vonalstílusok).
			\end{itemize}
		\end{block}
	\end{frame}
	
	\begin{frame}{dplyr csomag}
		A \texttt{dplyr} csomag megkönnyíti az adatok átalakítását és feldolgozását, különösen nagyobb adatállományok esetén.
		\vfill
		\begin{block}{Miért használtam}
			\begin{itemize}
				\item Segít az adatok gyors és egyszerű átalakításában.
				\item A mintaátlagok kiszámításánál és az adatok előkészítésénél használtam.
				\item Könnyen kombinálható más csomagokkal, például a \texttt{ggplot2}-vel.
			\end{itemize}
		\end{block}
	\end{frame}
	
	\section{Szimuláció R-ben}
	\begin{frame}{A szimulációk felépítése}
		\begin{itemize}
			\item Véletlen számok generálása különböző eloszlásokból.
			\item Mintavételezés és átlagok kiszámítása.
			\item Mintaátlagok ábrázolása hisztogram formájában.
			\item Empirikus sűrűség és normális eloszlás sűrűségfüggvény ábrázolása összehasonlításképpen.
		\end{itemize}
		\begin{block}{Megjegyzés}
			Az alábbi 3 példa futtatható a demo (1) opció kiválasztásával, viszont a felhasználó konzolos bemenetről tetszőleges paraméterekkel is indíthat szimulációt a (2) opcióval. A jobb oldalon látható képeket a program generálta.
		\end{block}
	\end{frame}
	
	\begin{frame}{Szimulációs példa 1 | Exponenciális eloszlás}
		\begin{columns}
			\begin{column}{0.5\textwidth}
				\textbf{Példa:} Szimuláljuk a CHT-t exponenciális eloszlású adatokkal.
				\begin{itemize}
					\item Generáljunk \textbf{5000} mintát exponenciális eloszlásból (\(\lambda = 1\)).
					\item Minden minta \textbf{100} elemű.
					\item Számítsuk ki minden minta átlagát, eloszlásait ábrázoljuk hisztogramon.
					\item Ábrázoljuk az empirikus sűrűséget és a normális eloszlást.
				\end{itemize}
			\end{column}
			\begin{column}{0.5\textwidth}
				\begin{figure}
					\includegraphics[width=\textwidth]{img/sim-1.png}
					\caption{CHT exponenciális eloszlás példa}
				\end{figure}
			\end{column}
		\end{columns}
	\end{frame}
	
	\begin{frame}{Szimulációs példa 2 | Binomiális eloszlás}
		\begin{columns}
			\begin{column}{0.5\textwidth}
				\textbf{Példa:} Szimuláljuk a CHT-t binomiális eloszlású adatokkal.
				\begin{itemize}
					\item Generáljunk \textbf{1000} mintát binomiális eloszlásból (\(n=10, p=0.3\)).
					\item Minden minta \textbf{100} elemű.
					\item Számítsuk ki minden minta átlagát, eloszlásait ábrázoljuk hisztogramon.
					\item Ábrázoljuk az empirikus sűrűséget és a normális eloszlást.
				\end{itemize}
			\end{column}
			\begin{column}{0.5\textwidth}
				\begin{figure}
					\includegraphics[width=\textwidth]{img/sim-2.png}
					\caption{CHT binomiális eloszlás példa}
				\end{figure}
			\end{column}
		\end{columns}
	\end{frame}
	
	\begin{frame}{Szimulációs példa 3 | Egyenletes eloszlás}
		\begin{columns}
			\begin{column}{0.5\textwidth}
				\textbf{Példa:} Szimuláljuk a CHT-t egyenletes eloszlású adatokkal.
				\begin{itemize}
					\item Generáljunk \textbf{5000} mintát egyenletes eloszlásból (\([0,1]\)).
					\item Minden minta \textbf{100} elemű.
					\item Számítsuk ki minden minta átlagát, eloszlásait ábrázoljuk hisztogramon.
					\item Ábrázoljuk az empirikus sűrűséget és a normális eloszlást.
				\end{itemize}
			\end{column}
			\begin{column}{0.5\textwidth}
				\begin{figure}
					\includegraphics[width=\textwidth]{img/sim-3.png}
					\caption{CHT egyenletes eloszlás példa}
				\end{figure}
			\end{column}
		\end{columns}
	\end{frame}
	
	
	\section{Köszönöm a figyelmet!}

	
\end{document}
