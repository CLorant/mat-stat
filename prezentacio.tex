\documentclass[aspectratio=169,12pt]{beamer}

\usepackage[magyar]{babel}
\usepackage{t1enc}
\usepackage{graphicx}
\usepackage{amsmath}

\definecolor{craneyellow}{RGB}{233, 187, 61}
\definecolor{craneblue}{RGB}{4, 6, 76}

\usetheme[progressbar=frametitle]{metropolis}
\usecolortheme{crane}
\usefonttheme{professionalfonts}

\hypersetup{
	colorlinks=true,
	linkcolor=craneblue,
	urlcolor=craneyellow
}

\title{Centrális Határeloszlás-tétel (CHT)}
\subtitle{Matematikai Statisztika}
\author{Czibik Lóránt Patrik}
\institute{Miskolci Egyetem}
\date{\today}

\begin{document}
	\begin{frame}
		\titlepage
	\end{frame}
	
	\section{Bevezetés}
	\begin{frame}{Bevezetés}
		\textbf{A Centrális Határeloszlás-tétel lényege:}
		\begin{itemize}
			\item Nagyméretű minták átlaga közelít a normális eloszláshoz.
			\item Fontos a statisztikában és a valószínűségszámításban.
			\item Gyakorlati alkalmazások a különböző tudományterületeken.
		\end{itemize}
	\end{frame}
	
	\section{Matematikai alapok}
	\begin{frame}{A Centrális Határeloszlás-tétel definíciója}
		\textbf{Definíció:} Legyenek $X_1, X_2, \ldots, X_n$ független, azonos eloszlású valószínűségi változók, melyek várható értéke $\mu$ és szórása $\sigma$. Ekkor a megfelelően normált összeg:
		\begin{equation}
			Z_n = \frac{\sum_{i=1}^{n} X_i - n\mu}{\sigma \sqrt{n}}
		\end{equation}
		$n \to \infty$ esetén standard normális eloszlást követ: $Z_n \sim N(0,1)$.
	\end{frame}
	
	
	\section{Alkalmazások}
	\begin{frame}{Gyakorlati alkalmazások}
		\begin{itemize}
			\item \textbf{Statisztikai következtetés:} A CHT alapvető szerepet játszik a mintavételi eloszlások és a hipotézisvizsgálatok területén.
			\item \textbf{Minőségellenőrzés:} Segítségével meghatározhatók a gyártási folyamatok során előforduló variabilitások.
			\item \textbf{Pénzügy:} A részvényhozamok és egyéb pénzügyi mutatók eloszlásának modellezésében is alkalmazzák.
		\end{itemize}
	\end{frame}
	
	\section{R nyelv és használt csomagok}
	\begin{frame}{Az R nyelv}
		Az R egy nyílt forráskódú nyelv statisztikai számításokhoz és adatelemzéshez.
		\vfill
		\begin{block}{Miért használtam}
			\begin{itemize}
				\item Erőteljes statisztikai és grafikai funkciók
				\item Könnyen bővíthető csomagokkal
				\item Korábbi tapasztalat
			\end{itemize}
		\end{block}
	\end{frame}
	
	\begin{frame}{ggplot2 csomag}
		A \texttt{ggplot2} csomag a grafikonok készítésére szolgál, pl. hisztogramok és sűrűségfüggvények.
		\vfill
		\begin{block}{Miért használtam}
			\begin{itemize}
				\item Segít a mintaátlagok és normális eloszlás ábrázolásában
				\item Testreszabható ábrák, pl. színek, jelmagyarázatok
			\end{itemize}
		\end{block}
	\end{frame}
	
	\begin{frame}{dplyr csomag}
		A \texttt{dplyr} csomag az adatok manipulálására (pl. data.frame) szolgál.
		\vfill
		\begin{block}{Miért használtam}
			\begin{itemize}
				\item Megkönnyíti a data frame kezelését és feldolgozását, ami szükséges a bővíthetőséghez
				\item Hasznos funkciók: szűrés (\texttt{filter}), rendezés (\texttt{arrange}), stb.
			\end{itemize}
		\end{block}
	\end{frame}
	
	\section{Szimuláció R-ben}
	\begin{frame}{A szimulációk felépítése}
		\begin{itemize}
			\item Véletlen számok generálása különböző eloszlásokból.
			\item Mintavételezés és átlagok kiszámítása.
			\item Mintaátlagok ábrázolása hisztogram formájában.
			\item Empirikus sűrűség és normális eloszlás ábrázolása összehasonlításképpen.
		\end{itemize}
		\begin{block}{Megjegyzés}
			Az alábbi 3 példa bele van hardcode-olva a programba, viszont bővíthető ha a \texttt{simulations} listához hozzáadunk elemeket.
		\end{block}
	\end{frame}
	
	\begin{frame}{Szimulációs példa 1 | Exponenciális eloszlás}
		\begin{columns}
			\begin{column}{0.5\textwidth}
				\textbf{Példa:} Szimuláljuk a CHT-t exponenciális eloszlású adatokkal.
				\begin{itemize}
					\item Generáljunk \textbf{5000} mintát exponenciális eloszlásból (\(\lambda = 1\)).
					\item Minden minta \textbf{100} elemű.
					\item Számítsuk ki minden minta átlagát, eloszlásait ábrázoljuk hisztogramon.
					\item Ábrázoljuk az empirikus sűrűséget és a normális eloszlást.
				\end{itemize}
			\end{column}
			\begin{column}{0.5\textwidth}
				\begin{figure}
					\includegraphics[width=\textwidth]{img/sim-1.png}
					\caption{A CHT exponenciális eloszlással szemléltetve}
				\end{figure}
			\end{column}
		\end{columns}
	\end{frame}
	
	\begin{frame}{Szimulációs példa 2 | Binomiális eloszlás}
		\begin{columns}
			\begin{column}{0.5\textwidth}
				\textbf{Példa:} Szimuláljuk a CHT-t binomiális eloszlású adatokkal.
				\begin{itemize}
					\item Generáljunk \textbf{1000} mintát binomiális eloszlásból (\(n=10, p=0.3\)).
					\item Minden minta \textbf{100} elemű.
					\item Számítsuk ki minden minta átlagát, eloszlásait ábrázoljuk hisztogramon.
					\item Ábrázoljuk az empirikus sűrűséget és a normális eloszlást.
				\end{itemize}
			\end{column}
			\begin{column}{0.5\textwidth}
				\begin{figure}
					\includegraphics[width=\textwidth]{img/sim-2.png}
					\caption{A CHT binomiális eloszlással szemléltetve}
				\end{figure}
			\end{column}
		\end{columns}
	\end{frame}
	
	\begin{frame}{Szimulációs példa 3 | Egyenletes eloszlás}
		\begin{columns}
			\begin{column}{0.5\textwidth}
				\textbf{Példa:} Szimuláljuk a CHT-t egyenletes eloszlású adatokkal.
				\begin{itemize}
					\item Generáljunk \textbf{5000} mintát egyenletes eloszlásból (\([0,1]\)).
					\item Minden minta \textbf{100} elemű.
					\item Számítsuk ki minden minta átlagát, eloszlásait ábrázoljuk hisztogramon.
					\item Ábrázoljuk az empirikus sűrűséget és a normális eloszlást.
				\end{itemize}
			\end{column}
			\begin{column}{0.5\textwidth}
				\begin{figure}
					\includegraphics[width=\textwidth]{img/sim-3.png}
					\caption{A CHT egyenletes eloszlással szemléltetve}
				\end{figure}
			\end{column}
		\end{columns}
	\end{frame}
	
	
	\section{Köszönöm a figyelmet!}

	
\end{document}
